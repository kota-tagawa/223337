\documentclass[]{jarticle}          % 一段組
%\documentclass[twocolumn]{jarticle} % 二段組

\textwidth 180mm
\textheight 255mm
\oddsidemargin -12mm
\topmargin -15mm
\columnsep 10mm

%\vspace{0.5cm} % 一段組の場合はコメントアウトした方が体裁がよいx
%] % 一段組の場合はコメントアウトする

\usepackage{styles/labheadings}
\usepackage[dvipdfmx]{graphicx,color}
\usepackage{amsmath,amssymb}
\usepackage{url}
% 追加
\usepackage[hang,small,bf]{caption}
\usepackage[subrefformat=parens]{subcaption}
\captionsetup{compatibility=false}

\newcommand{\aU}{\mbox{\boldmath $a$}}
\newcommand{\bU}{\mbox{\boldmath $b$}}
\newcommand{\cU}{\mbox{\boldmath $c$}}
\newcommand{\dU}{\mbox{\boldmath $d$}}
\newcommand{\eU}{\mbox{\boldmath $e$}}
\newcommand{\fU}{\mbox{\boldmath $f$}}
\newcommand{\gU}{\mbox{\boldmath $g$}}
\newcommand{\hU}{\mbox{\boldmath $h$}}
\newcommand{\iU}{\mbox{\boldmath $i$}}
\newcommand{\jU}{\mbox{\boldmath $j$}}
\newcommand{\kU}{\mbox{\boldmath $k$}}
\newcommand{\lU}{\mbox{\boldmath $l$}}
\newcommand{\mU}{\mbox{\boldmath $m$}}
\newcommand{\nU}{\mbox{\boldmath $n$}}
\newcommand{\oU}{\mbox{\boldmath $o$}}
\newcommand{\pU}{\mbox{\boldmath $p$}}
\newcommand{\qU}{\mbox{\boldmath $q$}}
\newcommand{\rU}{\mbox{\boldmath $r$}}
\newcommand{\sU}{\mbox{\boldmath $s$}}
\newcommand{\tU}{\mbox{\boldmath $t$}}
\newcommand{\uU}{\mbox{\boldmath $u$}}
\newcommand{\vU}{\mbox{\boldmath $v$}}
\newcommand{\wU}{\mbox{\boldmath $w$}}
\newcommand{\xU}{\mbox{\boldmath $x$}}
\newcommand{\yU}{\mbox{\boldmath $y$}}
\newcommand{\zU}{\mbox{\boldmath $z$}}
\newcommand{\AU}{\mbox{\boldmath $A$}}
\newcommand{\BU}{\mbox{\boldmath $B$}}
\newcommand{\CU}{\mbox{\boldmath $C$}}
\newcommand{\DU}{\mbox{\boldmath $D$}}
\newcommand{\EU}{\mbox{\boldmath $E$}}
\newcommand{\FU}{\mbox{\boldmath $F$}}
\newcommand{\GU}{\mbox{\boldmath $G$}}
\newcommand{\HU}{\mbox{\boldmath $H$}}
\newcommand{\IU}{\mbox{\boldmath $I$}}
\newcommand{\JU}{\mbox{\boldmath $J$}}
\newcommand{\KU}{\mbox{\boldmath $K$}}
\newcommand{\LU}{\mbox{\boldmath $L$}}
\newcommand{\MU}{\mbox{\boldmath $M$}}
\newcommand{\NU}{\mbox{\boldmath $N$}}
\newcommand{\OU}{\mbox{\boldmath $O$}}
\newcommand{\PU}{\mbox{\boldmath $P$}}
\newcommand{\QU}{\mbox{\boldmath $Q$}}
\newcommand{\RU}{\mbox{\boldmath $R$}}
\newcommand{\SU}{\mbox{\boldmath $S$}}
\newcommand{\TU}{\mbox{\boldmath $T$}}
\newcommand{\UU}{\mbox{\boldmath $U$}}
\newcommand{\VU}{\mbox{\boldmath $V$}}
\newcommand{\WU}{\mbox{\boldmath $W$}}
\newcommand{\XU}{\mbox{\boldmath $X$}}
\newcommand{\YU}{\mbox{\boldmath $Y$}}
\newcommand{\ZU}{\mbox{\boldmath $Z$}}
\newcommand{\epU}{\mbox{\boldmath $\epsilon$}}
\newcommand{\taU}{\mbox{\boldmath $\tau$}}
\newcommand{\etU}{\mbox{\boldmath $\eta$}}
\newcommand{\xiU}{\mbox{\boldmath $\xi$}}
\newcommand{\wwU}{\mbox{\boldmath $\omega$}}
\newcommand{\WwU}{\mbox{\boldmath $\Omega$}}
\newcommand{\lmU}{\mbox{\boldmath $\lambda$}}
\newcommand{\LmU}{\mbox{\boldmath $\Lambda$}}
\newcommand{\PiU}{\mbox{\boldmath $\Pi$}}
\newcommand{\SgU}{\mbox{\boldmath $\Sigma$}}
\newcommand{\thU}{\mbox{\boldmath $\theta$}}
\newcommand{\ThU}{\mbox{\boldmath $\Theta$}}
\newcommand{\roU}{\mbox{\boldmath $\rho$}}
\newcommand{\nuU}{\mbox{\boldmath $\nu$}}
\newcommand{\ones}{{\bf 1}}
\newcommand{\zr}{{\bf 0}}
\newcommand{\eq}{\begin{equation}}
\newcommand{\en}{\end{equation}}
\newcommand{\eqa}{\begin{eqnarray}}
\newcommand{\ena}{\end{eqnarray}}
\newcommand{\xx}{\makebox[1cm]{}}
\newcommand{\xm}{\makebox[0.5cm]{}}
\newcommand{\x}{\makebox[0.2cm]{}}
\newcommand{\tr}{{\rm tr}}
\newcommand{\sgn}{{\rm sgn}}
\newcommand{\ad}{{\rm ad}}

\newcommand{\rank}{{\rm rank}}
\newcommand{\diag}{{\rm diag}}
\newcommand{\lbr}{\left(\begin{array}}
\newcommand{\rbr}{\end{array}\right)}
\newcommand{\Proof}{\noindent{\em Proof\/}}
\newcommand{\Solution}{\noindent{\em Solution}}
\newcommand{\Derivation}{\noindent{\em Derivation}}
\newcommand{\msp}{\vspace*{\medskipamount}\\}
\newcommand{\qed}{\hspace*{\fill}$\Box$}
\newcommand{\aX}{{\bf a}}
\newcommand{\bX}{{\bf b}}
\newcommand{\cX}{{\bf c}}
\newcommand{\dX}{{\bf d}}
\newcommand{\eX}{{\bf e}}
\newcommand{\fX}{{\bf f}}
\newcommand{\gX}{{\bf g}}
\newcommand{\hX}{{\bf h}}
\newcommand{\iX}{{\bf i}}
\newcommand{\jX}{{\bf j}}
\newcommand{\kX}{{\bf k}}
\newcommand{\lX}{{\bf l}}
\newcommand{\mX}{{\bf m}}
\newcommand{\nX}{{\bf n}}
\newcommand{\oX}{{\bf o}}
\newcommand{\pX}{{\bf p}}
\newcommand{\qX}{{\bf q}}
\newcommand{\rX}{{\bf r}}
\newcommand{\sX}{{\bf s}}
\newcommand{\tX}{{\bf t}}
\newcommand{\uX}{{\bf u}}
\newcommand{\vX}{{\bf v}}
\newcommand{\wX}{{\bf w}}
\newcommand{\xX}{{\bf x}}
\newcommand{\yX}{{\bf y}}
\newcommand{\zX}{{\bf z}}
\newcommand{\AX}{{\bf A}}
\newcommand{\BX}{{\bf B}}
\newcommand{\CX}{{\bf C}}
\newcommand{\DX}{{\bf D}}
\newcommand{\EX}{{\bf E}}
\newcommand{\FX}{{\bf F}}
\newcommand{\GX}{{\bf G}}
\newcommand{\HX}{{\bf H}}
\newcommand{\IX}{{\bf I}}
\newcommand{\JX}{{\bf J}}
\newcommand{\KX}{{\bf K}}
\newcommand{\LX}{{\bf L}}
\newcommand{\MX}{{\bf M}}
\newcommand{\NX}{{\bf N}}
\newcommand{\OX}{{\bf O}}
\newcommand{\PX}{{\bf P}}
\newcommand{\QX}{{\bf Q}}
\newcommand{\RX}{{\bf R}}
\newcommand{\SX}{{\bf S}}
\newcommand{\TX}{{\bf T}}
\newcommand{\UX}{{\bf U}}
\newcommand{\VX}{{\bf V}}
\newcommand{\WX}{{\bf W}}
\newcommand{\XX}{{\bf X}}
\newcommand{\YX}{{\bf Y}}
\newcommand{\ZX}{{\bf Z}}

% report.texと同じディレクトリにnumerical_definition.texを入れておけば上の書き方でもいいはずです

\usepackage[
  dvipdfm,
  bookmarks=true,
  bookmarksnumbered=true,
  colorlinks=true]{hyperref}
\AtBeginDvi{\special{pdf:tounicode EUC-UCS2}}

\pagestyle{labheadings}
\headerleft{第1回レポート課題}   % ヘッダの左側のタイトル
\headerright{2024年6月19日}  % ヘッダの右側のタイトル

\begin{document}

%\twocolumn % 一段組の場合はコメントアウトする

\vspace*{2ex}
\begin{center}
 {\Large \bf 2023年度 「 シミュレーション特論 」第1回レポート課題}\\ % タイトル
 \vspace*{5mm}
 {\large M1 田川幸汰}% 発表者名
\end{center}

%\vspace{0.5cm} % 一段組の場合はコメントアウトした方が体裁がよいx
%] % 一段組の場合はコメントアウトする

%新しく作成したコマンド
% \newcommand{\reffig}[1]{\hyperref[#1]{図\ref{#1}}}
% \newcommand{\refeq}[1]{\hyperref[#1]{式(\ref{#1})}}
% \newcommand{\reftab}[1]{\hyperref[#1]{表\ref{#1}}}
% \newcommand{\refsec}[1]{\hyperref[#1]{\ref{#1}章}}
% \newcommand{\refsubsec}[1]{\hyperref[#1]{\ref{#1}節}}

% 数式
%\begin{equation}
%  数式記述  
%  \label{ラベル名}
%\end{equation}

% 図
% \begin{figure}[!ht]
%   \begin{center}
%     \includegraphics[scale=0.5]{figures/画像ファイル名}
%     \caption{キャプション名}
%     \label{ラベル名}
%   \end{center}
% \end{figure}

% リスト
% \begin{enumerate or itemize}
%   \item 
% \end{enumerate or itemize}

\section{課題1}
姫野ベンチのOpenMP並列版ソースコード( C + OMP, dynamic allocate version) をダウンロード、コン
パイルしOpenMPスレッド数を1としてベンチマークを実行せよ。
\subsection{解答}
スレッド数を1に固定し、計算サイズをS,M,Lと変化させた場合の
実行結果を\hyperref[one]{表\ref{one}}に示す。
\begin{table}[ht!]
  \begin{center}
    \begin{tabular}{lrrr}
      Grid size & CPU time (sec.) & Loop executed & MFLOPS \\
      S & 50.793346 & 7143 & 2315.829962 \\
      M & 58.208158 & 862 & 2030.379096 \\
      L & 55.346954 & 94 & 1899.987113 \\
    \end{tabular}
    \caption{ベンチーマーク結果}
    \label{one}
  \end{center}
\end{table}
\\
予想ではグリッドサイズを増やすことでループ実行回数が増えると考えていたが、
\hyperref[one]{表\ref{one}}より、グリッドサイズを増加すると、ループ実行回数が大きく減少していることがわかる。
これは、グリッドサイズを大きくすることでループ一回分の計算量が増え、
ループ実行回数の減少につながってリるのではないかと考えた。
また、実行時間はやや増加し、計算効率はやや減少した。


\section{課題2}
姫野ベンチのOpenMP並列版にてグリッドサイズをSとしてベンチマークを実行せよ。
\subsection{解答}
グリッドサイズを1に固定し、スレッド数を1,2,4と変化させた場合の
実行結果を\hyperref[two]{表\ref{two}}に示す。
\begin{table}[ht!]
  \begin{center}
    \begin{tabular}{lrrr}
      OpenMP threads num & CPU time (sec.) & Loop executed & MFLOPS \\
      1 & 50.793346 & 7143 & 2315.829962 \\
      2 & 37.366718 & 10090 & 4446.712517 \\
      4 & 34.667589 & 17919 & 8511.830593 \\
    \end{tabular}
    \caption{ベンチーマーク結果}
    \label{two}
  \end{center}
\end{table}
\\
予想ではスレッド数を増やすことでループ実行回数が減ると考えていたが、
\hyperref[two]{表\ref{two}}より、スレッド数を増加すると、ループ実行回数が大きく増加していることがわかる。
これは、スレッド数を大きくすることでループ一回分の計算量が減り、
ループ実行回数の増加につながってリるのではないかと考えた。
また、実行時間はやや減少し、計算効率はやや増加した。
実行時間と計算効率は反比例し、グリッドサイズを増加させた場合とは反対の結果が得られることがわかった。

\section{実行環境}
実行環境を以下に示す。
\begin{itemize}
  \item 計算ノード : xsnd00
  \item CPU名 : Intel Xeon Gold 6148 CPU @ 2.40GHz
  \item メモリ量 : 6291456 [kb]
  \item コンパイル時のコマンド : icx -qopenmp himenoBMTxpa.c -o himenoBMTxpa.x
\end{itemize}

\end{document}
