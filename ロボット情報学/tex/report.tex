\documentclass[]{jarticle}          % 一段組
%\documentclass[twocolumn]{jarticle} % 二段組

\textwidth 180mm
\textheight 255mm
\oddsidemargin -12mm
\topmargin -15mm
\columnsep 10mm

%\vspace{0.5cm} % 一段組の場合はコメントアウトした方が体裁がよいx
%] % 一段組の場合はコメントアウトする

\usepackage{styles/labheadings}
\usepackage[dvipdfmx]{graphicx,color}
\usepackage{amsmath,amssymb}
\usepackage{url}
% 追加
\usepackage{listings,jvlisting} 
\usepackage[hang,small,bf]{caption}
\usepackage[subrefformat=parens]{subcaption}
\usepackage{indentfirst}
\captionsetup{compatibility=false}

\input{numerical_definition.tex}
% report.texと同じディレクトリにnumerical_definition.texを入れておけば上の書き方でもいいはずです

\usepackage[
  dvipdfm,
  bookmarks=true,
  bookmarksnumbered=true,
  colorlinks=true]{hyperref}
\AtBeginDvi{\special{pdf:tounicode EUC-UCS2}}

%ここからソースコードの表示に関する設定
\lstset{
  basicstyle={\ttfamily},
  identifierstyle={\small},
  commentstyle={\smallitshape},
  keywordstyle={\small\bfseries},
  ndkeywordstyle={\small},
  stringstyle={\small\ttfamily},
  frame={tb},
  breaklines=true,
  columns=[l]{fullflexible},
  numbers=left,
  xrightmargin=0zw,
  xleftmargin=3zw,
  numberstyle={\scriptsize},
  stepnumber=1,
  numbersep=1zw,
  lineskip=-0.5ex
}
%ここまでソースコードの表示に関する設定

\pagestyle{labheadings}
\headerleft{最終課題}   % ヘッダの左側のタイトル
\headerright{2024年5月15日}  % ヘッダの右側のタイトル

\begin{document}

%\twocolumn % 一段組の場合はコメントアウトする

\vspace*{2ex}
\begin{center}
 {\Large \bf ロボット情報学特論期末レポート}\\ % タイトル
 \vspace*{5mm}
 {\large M1 田川幸汰}% 発表者名
\end{center}

%\vspace{0.5cm} % 一段組の場合はコメントアウトした方が体裁がよいx
%] % 一段組の場合はコメントアウトする

%新しく作成したコマンド
% \newcommand{\reffig}[1]{\hyperref[#1]{図\ref{#1}}}
% \newcommand{\refeq}[1]{\hyperref[#1]{式(\ref{#1})}}
% \newcommand{\reftab}[1]{\hyperref[#1]{表\ref{#1}}}
% \newcommand{\refsec}[1]{\hyperref[#1]{\ref{#1}章}}
% \newcommand{\refsubsec}[1]{\hyperref[#1]{\ref{#1}節}}

\section{課題1}
第1回から第7回の講義を通して、学んだことなどを整理する。

第1回ではコンヴィヴィアリティのためのヒューマンロボットインターフェース(HRI)デザインという内容について議論する。
ここでは、コンヴィヴィアリティという概念を人とロボットの関係に当てはめる。
コンヴィヴィアリティはここでは人とロボットがが協力して共に生活し、互いに支え合いながら豊かな関係を築くことを指す。
人とロボットの関係の例として、ベッドから車椅子への移乗を手伝ってくれるロボットがある。
ロボットは人を移乗するという操作を完遂し、人はロボットの荷物のような存在となる。
このように自己完結した"強いロボット"は人間の主体性を奪い、人の自立性や創造性を奪ってしまう側面がある。
コンヴィヴィアリティという概念を人とロボットの関係に当てはめた例として猫の顔をした配膳ロボットがある。
ロボットはまだ拙い所作ながらホールのなかをトコトコとうごきまわり、人は配膳された食事をテーブルに写すというかたちで作業の完遂をサポートする。
このような"弱いロボット"は人と共に1つのシステム(we-mode)を作り上げることで、互いの弱みを補い合い、強みを引き出しあう(well-being)。
このように、HRIデザインではコンヴィヴィアリティの概念を意識する必要があり、このようなデザインを
コンヴィヴィアル・ロボティクスという。

第2回では認知的ロボティクスと状況論的認知という内容について議論する。
ここでは、ロボットの生き物らしさを生みだすものとして、状況論的な認知があげられている。
ダニエル・デネットは志向姿勢に関する議論で、私たちが目の前の対象の動きをみたとき、
物理的な構え、設計的な構え、志向的な構えのいずれかで捉えようとすると表した。
認知的ロボティクスの発展において、ロボットがどのように知覚し、行動するかという、志向的な構えで理解する必要がある。
また、デネットの志向姿勢は状況論的認知とも関連している。
状況論的認知は、知性や行動を対称の個体に帰属させる、という個体能力主義的な能力感とは異なり、
知性や行動が環境との相互作用によって生まれる、という考え方である。
このように、ロボットにおいても個体能力主義的な考えにとらわれず、環境との相互作用において行動を生みだす状況論的認知に基づく
アプローチで設計することができる。

第3回ではソーシャルな環境に埋め込まれるという内容について議論する。
わたしたちの身体は外側から容易に観察できることから、個体として完結しているような先入観を持たれやすい。
しかし自分の身体を内側から見ることはできないことから、実際にはわたしたちの身体は不完結で、それは外に開いていると考えることができる。
ここで、生態心理学者ジェームズ・ギブソンは、わたしたちの個体と私たちを取り囲んでいる環境との相互作用の結果として
価値ある事態を生み出しているのではないかと考えた。
ロボットについても、不完結であることを前提に、どのように他に開きながら価値ある事態を生み出すかというアプローチで考える。
例として、ゴミ拾いロボットは一人ではゴミを拾うことができない。
ゴミの近くに動き、他者にゴミを拾ってもらえるように助けを求めることで他者の手助けを上手に引き出し、目的を一緒に達成することができる。
このようにソーシャルな環境にあるロボットは、どこまでも自己完結、完全無欠を目指す個体能力主義的なアプローチだけでなく、
「不完全なもの」であることを認め、お互いにゆるく依存しあう関係論的なアプローチの視点で考えることができる。
また関係論的なアプローチでは、ロボットの能力や機能の「隙間を埋める」足し算としてのデザインでなく、
他者の参加する「隙間を生み出す」引き算としてのデザインをどのように考えるかが重要にある。

第4回では発話やコミュニケーションにおける関係論的な方略について議論する。
わたしたちはコミュニケーションをロボットとの距離感によって選んでいる。
例として、自動販売機の「ありがとう」がお礼の気持ちと結びつかないのは、
わたしたちが自動販売機に対して設計的な構えを取るためからだといえる。
ここで、前章で説明した関係論的アプローチを採用したロボットとのコミュニケーションを考える。
「トーキング・アイ」という仮想的な生き物たちと雑談するインタラクティブでは、
自己完結した言葉でなく、あえて不簡潔な言葉を使い、発話の意味の一部を相手に委ねることで、関係論的な行為方略が選ばれるようになる。
発話やコミュニケーションにおける関係論的な方略では、言葉の内容を相手に正確に伝える伝達的機能だけでなく、
言葉を相手との間でどのような新しい意味を生み出すかという意味の生成的機能を含んでいる。
このほかにも発話やコミュニケーションにおける関係論的な方略を採用した例をいくつか示す。
「トーキング・アリー」では、自然な発話には言い淀み、言い直しなどの非流暢な発話を多く含むことに注目し、これを採用することで、
なにかを伝えたいという意思を感じてしまい、思わずロボットに寄り添ってしまう。
「トーキング・ボーン」では、人とロボットの間に忘却という<媒介物>を用意することで、
コミュニケーションの中で三項関係をつくり、コミュニケーションにおける障壁を乗り越えようとする。

\end{document}
